\documentclass[a4paper,12pt]{article}

\title{Firstaidkit Explained}
\author{Joel Andres Granados, Martin Sivak}

\begin{document}
\maketitle

\section{Use Case}
Lets meet Joe.  He has a very healthy interest in Linux based Os's and has been a user of fedora for some months now.  He has a very positive experience with his currently installed fedora 9, which was installed by his hacker friend named Fred.  Fred, being a very knowledgeable hacker and being a very good friend of Joe's, installed a stable fedora system that made Joe very happy.  He has had very little issues with fedora 9, and the system has allowed him to get his daily job done.  Joe currently studies in his hometown University and majors in math.

Sometime after receiving fedora 9 from his hacker friend, Joe gets the word that fedora 10 was released and that it has a bunch of updates that might be interesting to him.  Being a curious and adventurous person by nature, Joe thinks its a good idea to upgrade his system, so he gets his hands on a fedora 10 DVD and fires up the upgrade process.  As usual its a very user friendly interface and after answering some configuration question that he does not completely understand, the upgrade finishes successfully.  Joe sees that the upgrade has ended and is eager to restart his laptop to begin to use all the wonderful updates that he had read about in fedora site.  So he reboots and waits impatiently for the system to come up again.

Success!, fedora has upgraded to its next major release.  Joe is happier than ever.  He is so happy that he begins to look for more information on the new release.  In all his searching he comes upon a site that tells him how to modify the background of his grub.  He has a really cool image that he thinks would look really nice at the grub menu.  He decides to follow the instructions on this particular site and see if he can make his cool idea work.  When the tutorial tells him to backup the first 512 bytes of his hard drive he gets mixed up and copies "dd of=/dev/zero of=/dev/sdb bs=1 count=512" on the command line, with root access.  He completes the rest of the tutorial and reboots when he thinks that he is done.

Joe waits a couple of minutes before realizing that the black screen with a white cursor at the top left is not the expected boot screen.  He feels scared because he does not have any other way of booting the system and all his work for the semester is in this particular computer.  He immediately thinks about all the important information that is contained inside and starts panicking.  He reboots the machine several times to find that the behavior does not change.  In a state of complete panic he calls Fred.  Fred's girlfriend picks up and says that Fred has gone out of the country and left his cell with her until he comes back.  But, Joe is not discouraged and he goes to a friends house to get on the web and try to find a way to fix his problem.  He spends the weekend trying to make his notebook boot with no success.

On the following Monday he gets help from the informatics office in his university.  They are quick to reinstall grub in his partition and make everything work again.  They explain to Joe what happened and teach him how to fix it in case it ever happened again.  Joe is somewhat happy because all his information is back where he left it, but he is left with a very negative experience from fedora/Linux.  It would not be surprising to consider that Joe will think twice next time he has to choose an OS for his personal use.  All this could have had a totally different outcome if there was a tool to fix common problems like the one that Joe had.  A tool that automatically diagnosed and fixed Joe's problems.  This is the type of tool we would like to present in this article.

\section{What is Firstaidkit}
Firstaidkit is an automated recovery tool that brings together common recovery processes and applies them to a system.  The way that Firstaidkit handles the recovery processes is by means of plugins.  The idea being that a plugin will focus on a particular issue in the system, like grub, init scripts or xserver.  Firstaidkit is designed to automatically fix problems while focusing on maintaining user data integrity.  In other words, Firstaidkit will try its best to fix your system while maintaining your data intact.

Here we can recall Joe's situation.  Lets recreate the end of the story with the only difference being that Joe had Firstaidkit on a rescue disk or liveCD.  Joe just wants his computer to boot, he knows that he somehow shot himself on the foot, but now he would want an easy way out.  In this case Joe may run Firstaidkit in diagnose mode to see what is wrong with his system.  Firstaidkit will notice the missing grub in the first 512 bytes of the drive and will tell Joe that he needs to fix his grub.  After looking through the Firstaidkit man pages and searching through the plugin list, Joe finds that he needs to run the plugin named "grub".  He tells Firstaidkit to run grub in fixing mode and reboots his machine after the process ends.  He waits impatiently for the machine to go past the blank screen with the cursor at the top left.  When the computer boots normally, Joe is very pleased that all his information was not lost and continues his daily activities.

Joe's situation is just one of many in which an automated recovery tool would make peoples life easier.  It is not restricted to the linux user that is a beginner and can be extended into administrator recovery tasks like a SELinux analysis process, deleted partition recovery process, init scripts recovery, rpm database recovery, etc.  The list extends into most, if not all, the OS subsystems.

But unfortunately Firstaidkit will not take care of everything.  Being a tool that executes common recovery processes, it needs some context from the existing system to actually get the fix right.  If you completely wipe out your hard drive it will be extremely difficult for Firstaidkit to get anything done.

\section{How does Firstaidkit work?}
The philosophy of Firstaidkit is to be a *fully* automated recovery system.  This means that once plugin execution has started there is no user interaction.  Everything must be specified in the configuration files, CLI or GUI.

Firstaidkit can be considered a list of recovery processes that used to recover default value behavior.  Most of the time the user will not be looking to do a full system check.  Instead a specific subsystem is identified as misbehaving and therefor targeted to be rescued.  The trick here is for the user to know how that misbehavior is referenced in Firstaidkit.  More specifically, what subsystem it relates to.  When a machine does not boot the failure can be in a lot of places, but to the user its one problem: My computer does not boot.  Each plugin will have a diagnose that will give the user important information about current state and hopefully point the user in a direction where he can solve the problem or find more information.  When the user runs a diagnose of the systems, messages like "Your initscripts are busted" and "You are missing a boot loader" will hint the user of the real root of the problem.  Moreover, if the user is using our GUI, he will be presented with red messages advising him of misbehaving subsystems.  This general diagnose can be the first step in the rescue process.

After the general diagnose the logical next step is to actually fix something.  Firstaidkit's plugin based framework give the user many ways of doing this.  Each plugin is a recover process in itself, they can be chosen to run individually, a group of plugins can be chosen to execute in one run, special parameters can be given to individual plugins, some plugins can have various fix methods while others can be restricted to a default behavior.  The plugin framework allows this diversity while still having a default behavior for the user that does not want to customize behavior.  Basically the user must specify the plugins to be run, the initial configuration values (they are specified in a config file) and that all the plugins are to be run in fixing mode.

After starting the plugin execution, Firstaidkit goes and does its magic.  While executing, it shows the user what is being done in the system by means of a log that is redirected to stdout or a file.  The GUI can also show the log in a very efficient way, coloring the warnings and errors in a matter in which the user can clearly see where stuff went wrong.  All this recollected information is important for the user to know what was done to the system and can be used in cases Firstaidkit is not able to fix the related subsystem.

Another nice thing about the plugin framework is the possibility of adding individual plugins from other sources.  Lets say your distribution does not provide the SELinux plugin that another distribution might have.  The only thing that needs to be done is to copy all the related plugin files or directories and point firstaidkit to the place where they reside.  Since they can be executed independently from each other there will be no interference if you run them individually.

Currently there are two ways of interacting with Firstaidkit: The Command Line Interface (CLI) and the Graphical User Interface (GUI).  Both of them have different ways of doing the same things, basically present the related information and provide means of executing plugins.  The GUI is the easiest way of interaction but also depends on lots of the subsystems, an X server being the most prominent.  But when the problem is not the X server, you can choose plugins, start execution, stop execution and modify various configuration variables from the GUI.  This way of interaction has a built in log filter that colors the warnings and errors accordingly to emphasize various states of the fixing process.  This is something that non experienced users will appreciate.  There is another downside to the GUI besides the dependency on the X server, its not scriptable.  There might be a situation where a sysadmin might want to create a general system diagnose script.  He may think it useful to run every time he does an update or does major changes on a system.  The GUI cannot aid the sysadmin in his task, but the CLI is a perfect fit for this kind of situation.  In summary the CLI and the GUI complement themselves and increase the user satisfaction for various types of users.

Some of Firstaidkit's behavior is controlled by ini style configuration files.  The system allows to pull any number of configuration files from various places.  It also allows the user to specify individual configuration values through the CLI and GUI.  This way of pulling configuration values from various places can get complicated, so Firstaidkit has the ability of printing the resulting configuration after parsing all the files.

At this point you might have guessed the importance of running Firstaidkit on a damaged system.  This is needed because Firstaidkit will probably be used in situations where a healthy running system is not an option.  There are two additional ways of getting to Firstaidkit:

\begin{itemize}
\item Rescue image:
To use the rescue functionality of the iso images is the most logical thing to do when your system is not well.  Specially if it does not boot.  Firstaidkit will be contained in the rescue image and can be accessed with the command line.  Note that there is no X in rescue mode.  This means that the only way of interacting with Firstaidkit is through CLI.
\item Live CD:
Live CD fits in really nice with the firstaidkit concept because it puts the user in a graphical environment without the need of a sane local system.  This gives Firstaidkit the opportunity to be in a place where the user can interact with its GUI.  The user can activate the recovery processes that he thinks are needed or just tell Firstaidkit to make a general scan of the system.
\end{itemize}

\section{Conclusion}
\begin{itemize}
\item Firstaidkit is an automated recovery tool that brings together common recovery processes.
\item Firstaidkit is based on a plugin backend.
\item Firstaidkit has a GUI and a CLI.
\item There are situations where having Firstaidkit is relevant.  Joe's problem just being one of many examples.  It can also be used by sysadmins as a diagnose and/or recovery tool.
\item It aids the user in precarious situations.  It ads to a good user experience.
\item It has no interaction with the user once it begins to execute.
\item It can be use to diagnose a system.  This diagnose can be performed by means of scripts and/or using the GUI.
\item When doing the diagnose and/or fixing, it will inform the user of relative information of the system and process being applied to the system.
\item The user can select individual plugins and execute them as preferred.
\item The user can execute Firstaidkit in a running system, in rescue mode, and on the liveCD.
\item The GUI increases the understanding of the log messages.
\item It is highly configurable and can handle more than one configuration source.
\item The inclusion of a plugin in the pool is as easy as copying the plugin file or directory.
\end{itemize}

\end{document}
